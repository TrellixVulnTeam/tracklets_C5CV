\chapter{Design and implementation}\label{chap:design}

	For implementation, we use two programming languages - Java and Python. The application is a console one, running firstly on Linux (see Chapter \ref{chap:requirements}, Section \ref{sec:server_param}), even though due to the languages chosen it runs on MS Windows without problems as well.
	
	We use several Java libraries and Python modules which allow for quicker and more efficient execution of many needed tasks.
	
	Java libraries:	
\begin{itemize}
	\item Commons Math: The Apache Commons Mathematics Library - used for all calculations related to SLR and other mathematical requirements,
	\item eap.fits - a NASA Java interface to FITS files, allowing for opening, reading and manipulating FITS files.
\end{itemize}		

	And Python modules:
\begin{itemize}
	\item Astropy - a module for performing astronomy and astrophysics related tasks in Python,
	\item tflearn - a deep learning higher-level library.
\end{itemize}

\section{Classes}\label{sec:classes}

	There are 12 Java classes (along with the Main class) in our implementation.
	
\subsection{algorithms.LinearRegression}

	A class responsible for handling SLR. It contains all user-set values denoting thresholds needed for fine-tuning accuracy of the SLR described in Chapter \ref{chap:proposed_solutions}, Section \ref{sec:linear_regression}.
	
\begin{table}[H]
\centering
\caption{Class variables table}
\setlength{\extrarowheight}{2pt}
\label{tab:class_variables_LR}
\begin{tabularx}{\textwidth}{|X|X|}
\hline
\textbf{Class variable} & \textbf{Description} \\ \hline
private static final double \mbox{distanceThreshold} & a double representing distance threshold \\ \hline
private static final double \mbox{angleThreshold} & a double representing angle threshold    \\ \hline
private static final double \mbox{speedThreshold} & a double representing speed threshold    \\ \hline
\end{tabularx}
\end{table}

\begin{table}[H]
\centering
\caption{Methods table}
\setlength{\extrarowheight}{2pt}
\label{tab:class_methods_LR}
\begin{tabularx}{\textwidth}{|X|X|}
\hline
\textbf{Method} & \textbf{Description} \\ \hline
public static void \mbox{perform} & a method responsible for performing all necessary actions, called from outside of the class \\ \hline
private static void \mbox{findInitialRegressions} & a method responsible for constructing initial regressions from starting points\\ \hline
private static void \mbox{fitPointsToRegressions} & a method responsible for handling all existing SLRs - modifying them, correcting them, etc.\\ \hline
\end{tabularx}
\end{table}

\subsection{algorithms.OrbitDetermination}

	A class wrapping already existing implementation of IOD. It handles translating information from previous steps and passing them to the IOD algorithms described in .
	
\begin{table}[H]
\centering
\caption{Class variables table}
\setlength{\extrarowheight}{2pt}
\label{tab:class_variables_OD}
\begin{tabularx}{\textwidth}{|X|X|}
\hline
\textbf{Class variable} & \textbf{Description} \\ \hline
private static final double \mbox{AGOLON} & a constant describing longitude of AGO \\ \hline
private static final double \mbox{AGOLAT} & a constant describing latitude of AGO\\ \hline
private static final double \mbox{AGOALT} & a constant describing altitude of AGO\\ \hline
\end{tabularx}
\end{table}

\begin{table}[H]
\centering
\caption{Methods table}
\setlength{\extrarowheight}{2pt}
\label{tab:class_methods_OD}
\begin{tabularx}{\textwidth}{|X|X|}
\hline
\textbf{Method} & \textbf{Description} \\ \hline
public static void \mbox{perform} & a method responsible for performing all necessary actions, called from outside of the class \\ \hline
private static void \mbox{transformObjectsIntoObservations} & a method responsible for translating inner objects into Observations (class in the existing implementation of IOD)\\ \hline
\end{tabularx}
\end{table}