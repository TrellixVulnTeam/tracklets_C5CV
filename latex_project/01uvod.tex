\chapter{Introduction}\label{chap:intro}

"Space debris" is a term which encompasses both natural and artificial objects and particles. While man-made objects, or "orbit debris", typically orbit around the Earth, meteoroids orbit around the sun and can have trajectories crossing that of the Earth. 

There are many risks accompanied by the existence of space debris, namely in-orbit collisions with operational spacecraft, and re-entries. Impacts by millimetre-sized debris could disable a subsystem of an operational spacecraft, debris larger than 1 centimetre would disable the whole spacecraft and debris larger than 10 centimetres would cause catastrophic break-ups. Due to this fact it is important to observe and catalogue debris to prevent collisions with functional man-made satellites and re-entries which could endanger populace.

The aim of this diploma thesis is to develop a comprehensive algorithm and analyse different possible approaches which would identify space debris and extract its position from astronomical CCD images containing observations of star background and unknown objects.