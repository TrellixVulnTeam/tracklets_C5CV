\chapter{Introduction}\label{chap:intro}
	
	While we are able to put vehicles, both manned and unmanned, into space, in majority of cases, it is not cost efficient to return it back to Earth. In some cases, such as unpredictable events, mistakes in calculations or even the effects of the Sun, cause orbiting satellites to collide or break up. Therefore, the term space debris encompasses all artificial non-functional objects orbiting around Earth.

	As is described in this document, there are almost no space debris creation mitigation processes in place and those which are, are too insignificant to make a visible difference. This results in the rapidly increasing number of space debris. Even though the major agencies, for example European Space Agency (ESA) or North American Space Agency (NASA), already have plans for major space debris clean-ups in motion, it is still vitally important to track, catalogue and observe existing debris, as it threatens missions, both future and present, and existing satellites.

	In this thesis, we analyse and provide another solution to the problem of assigning observations of the same object in time together - to the problem of tracklet building.