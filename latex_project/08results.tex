\chapter{Results}\label{chap:results}

\section{Bern and MPC formats}\label{sec:tdm_ccsds}

	This section provides a short insight over standardised output data formats used in this thesis and encouraged by multinational agencies and foreign universities. 
		
\subsection{MPC}\label{sec:mpc}

	Another worldwide organization for Astronomical purposes is the Minor Planet Center (MPC). The organization is in charge of collecting observational data for minor planets, comets and satellites of major planets and belongs under International Astronomical Union \citep{mpc}.
	
\subsection{MPC format}

	The officially MPC endorsed format is a text file with exactly prescribed form. The format is divided into columns where each column equals one character in a text file. The columns are explicitly set, therefore the format does not contain any headers. There are four kinds of formats - for minor planets, for comets, for natural satellites and for minor planets, comets and natural satellites. In our case, we are using the last type \citep{mpc}.
	
	The designation for every column is as follows:
	
	\begin{itemize}
		\item columns 1-12 -- designation of the observation
		\item column 14 -- NOTE 1 - an alphabetical publishable note (for example \emph{S} for "poor sky", or \emph{K} for "stacked image"
		\item column 15 -- NOTE 2 - indicates how observation has been made (for example \emph{C} for "CCD")
		\item columns 16-32 -- DATE OF OBSERVATIONS - contains the date and time in UTC of the mid-point of observation; the format of this column is "YYYY MM DD.dddddd"
		\item columns 33-44 -- OBSERVED RA - contains observed right ascension; the format of this column is "HH MM SS.ddd"
		\item columns 45-56 -- OBSERVED DECL - contains observed declination; the format of this column is "sDD MM SS.dd" ("s" for sign)
		\item columns 66-71 -- OBSERVED MAGNITUDE AND BAND - the observed magnitude of an object and band in which the measurement was made
		\item columns 78-80 -- OBSERVATORY CODE - the code of the observatory in which observation was made (AGO has code 118)
	\end{itemize}

\section{Visualization of results}\label{sec:visualization}
